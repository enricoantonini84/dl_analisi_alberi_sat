\chapter{Introduzione}
Questa ricerca ha l'obiettivo di analizzare e costruire un metodo per identificare la presenza di alberi e aree verdi dalle immagini satellitari delle aree urbane, utilizzando modelli di machine learning.
\\\\Negli ultimi anni, infatti, l'avvento dell'intelligenza artificiale ha cominciato a cambiare il modo in cui lavoriamo e studiamo, permettendoci di effettuare ricerche e generare anche media audio/video a partire da un semplice testo digitato in un prompt. Questi tipi di intelligenza artificiale tuttavia, basati su modelli LLM per quanto molto rivoluzionari, sono general purpose e lavorano sul linguaggio naturale.
\\\\L'idea è quella di utilizzare dei modelli di machine learning che, sfruttando un algoritmo di visione artificiale, siano in grado di identificare velocemente la presenza di oggetti, come le chiome o la forma degli alberi, per scopi di pianificazione urbanistica o per alimentare indicatori di accesso al verde, come quello determinato dalla regola del 3-30-300.

