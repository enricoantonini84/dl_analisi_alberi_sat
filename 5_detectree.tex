\chapter{DetecTree}
DetecTree è un algoritmo di machine learning che non si limita all’individuazione dei singoli alberi, ma consente anche di identificare intere aree boschive o zone verdi all’interno di immagini satellitari. Come introdotto nel capitolo tecnico in precedenza, non è catalogabile come rete neurale, in quanto l'algoritmo basato sul detector AdaBoost non è composto da nodi e layers ma "semplicemente" da un albero decisionale.\\

\section{Principio di funzionamento}
Il principio di funzionamento dell'algoritmo DetecTree è riassumibile nei seguenti punti \cite{bosch2020detectree}:
\begin{itemize}
    \item Suddivisione in tile: l’immagine aerea viene suddivisa in tile più piccole, per generare un mosaico di "tessere" di dimensione specifica.
    \item Scelta delle tile di training: vengono selezionati i tile da usare per l’addestramento tramite dei descrittori GIST, ovvero array numerici che sintetizzano caratteristiche visive e semantiche di una porzione d'immagine.
    \item Ground truth masks: per ciascun tile scelto per il training, si devono fornire delle maschere binarie “albero/non-albero”, con strumenti di editing delle immagini.
    \item Training: per ogni pixel, viene estratto un array di 27 feature e si addestra un classificatore AdaBoost, che è a tutti gli effetti un classificatore binario, che associa il vettore di feature alle classi "albero/non-albero".
    \item Testing: procedendo con l'inferenza sui tile di test, la classificazione viene poi raffinata con un algoritmo, migliorando la detection su pixel adiacenti classificati come albero.
\end{itemize}

\noindent
L'insieme delle 27 feature è una rappresentazione che viene data ad ogni singolo pixel, dal classificatore AdaBoost in fase di training, per permettere al modello di distinguere i pixel appartenenti o non appartenenti ad un albero. Questo tipo di classificazione viene eseguita in base sia alle informazioni cromatiche che al contesto dell'immagine.\\
Le feature si distinguono in:

\begin{itemize}
    \item 6 di colore: contengono dati sul colore del pixel, come ad esempio ai canali RGB e possibili combinazioni derivate.
    \item 18 di texture: sono misure statistiche (es. media, varianza, contrasto, ecc.) calcolate su regioni locali attorno al pixel, spesso derivate da modelli.
    \item 3 di entropia: misura l’imprevedibilità dei valori di intensità in una specifica regione attorno al pixel.
\end{itemize}

\section{Tipologia di modello utilizzato}
In questo caso la repository del progetto github mette a disposizione il modello pre-trained, in modo da poter essere utilizzata direttamente senza dover ricorrere alla procedura di addestramento ma, soprattutto, senza dover ricorrere alla annotazione manuale di immagini per il dataset di training.

\section{Parametri}
Nel caso di DetecTree, oltre all’immagine che evidenzia la copertura arborea, è disponibile anche il valore percentuale della copertura presente all’interno dell'immagine analizzata.

\section{Inferenza}
\begin{lstlisting}
    import skops.io as sio

    untrusted_types = sio.get_untrusted_types(file="detectree_model.skops")
    model = sio.load("detectree_model.skops", trusted=untrusted_types)

    clf = dtr.Classifier(clf=model)
    pred = clf.predict_img(image_path)
\end{lstlisting}

\noindent
Queste cinque righe di codice sono responsabili del caricamento del modello e dell'inferenza con DetecTree.\\
Per poter classificare l'immagine, è necessario prima di tutto importare la libreria Skops \cite{skops}, che è una libreria Python utilizzata per importare i modelli basati su Scikit-learn, celebre piattaforma open source per il machine learning in linguaggio Python.\\
Il modello sarebbe potenzialmente importabile direttamente da Scikit, ma non è consigliabile procedere in tal senso, in quanto la libreria sfrutta un modulo standard di Python (Pickle) per serializzare e deserializzare i modelli, che può eseguire codice arbitrario durante la deserializzazione, esponendo a potenziali attacchi l'host sul quale è in esecuzione.\\
Nelle righe successive vengono caricati i types del modello che definiscono quale struttura o classe ha l'oggetto (come ad esempio un modello, un vettore o un dato correlato). In questo caso viene scelto di caricare tutti i types, anche quelli che non sono automaticamente considerati sicuri per essere caricati: anche caricando tipi considerati "non verificati", rimane comunque la protezione che Skops offre contro gli attacchi di deserializzazione di Pickle.\\
Caricato il modello si passa alla classificazione, eseguendo la predizione con \textit{predict\_img}, che restituisce la mappa di etichette (ad es. array 2D di classi) per l’immagine di input.

\section{Falsi positivi}
L'utilizzo di AdaBoost introduce comunque una criticità non trascurabile in quanto, lavorando per similarità, non solo è poco preciso a distinguere tra alberi o prati, ma è fortemente penalizzato nel caso in cui le aree verdi sono vicine ad altre zone con caratteristiche simili, come ad esempio i corsi d'acqua.\\
In questo caso, infatti, DetecTree fatica a delimitare i confini delle aree verdi, andando a classificare erroneamente aree che non dovrebbe considerare come boschive.\\
Un esempio significativo di questo problema è riscontrabile nell'area di ponte Aleardi a Verona, nei pressi dei giardini del cimitero monumentale: l'algoritmo classifica erroneamente come area arborea una porzione delle acque del fiume Adige, a causa della vicinanza cromatica con le chiome degli alberi di Lungadige Porta Vittoria.